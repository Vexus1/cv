\documentclass[10pt]{article}
\usepackage{polski}
\usepackage[left=1cm, right=1.5cm, top=0cm, bottom=1cm]{geometry}
\usepackage{multicol}
% \renewcommand{\familydefault}{\sfdefault}
\usepackage{lipsum}
\usepackage{graphicx}
\usepackage[T1]{fontenc}
\usepackage{charter}
\usepackage{enumitem}
\usepackage{fancyvrb}
\usepackage{nopageno}
\usepackage[dvipsnames]{xcolor}
\usepackage{paracol} 

\begin{document}

% -- BANER tylko na pierwszej stronie --
\begin{figure}[t]
    \hskip-1.55cm\includegraphics[scale=1.3625]{Galik_BANNER.png}
\end{figure}

% Rozpoczynamy środowisko paracol z 2 kolumnami:
\columnratio{0.3,0.7}
\setlength{\columnsep}{2cm}  
\begin{paracol}{2}
% Ustawiamy proporcje szerokości kolumn (np. 0.3 : 0.7)

%------------------- LEWA KOLUMNA -------------------%
\begin{leftcolumn}
    \vspace*{-2ex} % ewentualne przesunięcie w górę, by tekst był bliżej bannera
    \noindent \textbf{+48 663 383 000 \\
    adrian1galik@gmail.com \\
    github.com/Vexus1} \\
    \rule{6cm}{1pt} \\ \\
    \fontsize{14pt}{14pt}{\textbf{\color{Violet}UMIEJĘTNOŚCI:}}
    \fontsize{10pt}{10pt}
    \begin{itemize}[leftmargin=*]
        \setlength{\parskip}{0pt}
        % \setlength\itemsep{0pt} 
        \item \textbf{Python} poziom zaawansowany (+4 lat)
        % \item \textbf{C++} poziom początkujący
        \item Biblioteki programistyczne: \raggedright \\ \textbf{NumPy, PyTorch, TensorFlow, Keras, Gymnasium, OpenCV, Scikit-Learn, Pandas, NetworkX}
        \item Zastosowania algorytmów głębokiego uczenia maszynowego
        \item Wysoka znajomość algorytmów uczenia przez wzmacnianie
        \item Abstrakcyjne struktury danych: \textbf{Stosy, Kolejki, Drzewa, Grafy}
        \item Zarządzanie bazami danych: \textbf{SQL}
        \item Modele i metody statystyki \\ matematczynej \textbf{język R}
        % \item Wizualizacja danych \textbf{Power BI}
        \item Tworzenie i administrowanie \\ stronami internetowymi: \textbf{HTML, CSS, JavaScript, React, Flask, PHP}
        \item Zastosowanie równań \\ różniczkowych
        % \item Projektowanie i zarządzanie \\ sieciami komputerowymi
        % \item Pakiety microsoft office: \textbf{Excel}
        \item System kontroli wersji: \textbf{Git}
        % \item Tworzenie i administrowanie \\ stronami internetowymi: \textbf{HTML, CSS, JavaScript, Flask, PHP}
        \item System operacyjny: \textbf{Linux, Windows}
        % \item Język formatowania tekstu: \textbf{LaTeX}
        \item Obliczenia numeryczne: \textbf{Julia}
        \item Powłoka systemowa UNIX: \textbf{Bash} 
        \item Środowisko tworzenia aplikacji: \textbf{Docker,} 
        % virtual box
        \item Framework wspierający rozwój oprogramowania dla robotów - \textbf{ROS2}
        \item Analityczne myślenie
        % \item Praca zespołowa 
    \end{itemize}

    \rule{6cm}{1pt} \\ \\
    \fontsize{14pt}{14pt}{\textbf{\color{Violet}JĘZYKI:}}
    \fontsize{10pt}{10pt}
    \begin{itemize}[leftmargin=*]
        \setlength{\parskip}{0pt}
        % \setlength\itemsep{0pt} 
        \item Polski ojczysty
        \item Angielski C1
        \item Hiszpański A1
    \end{itemize}
    \vspace{1500pt}
    \rule{0pt}{0pt} \\ \\ \\
    \fontsize{14pt}{14pt}{\color{Violet}\textbf{ZAINTERESOWANIA:}}
    \fontsize{10pt}{10pt}
    \begin{itemize}[leftmargin=*]
        \setlength{\parskip}{0pt}
        \item Uczenie maszynowe
        \item Matematyka
        \item Astrofizyka
    \end{itemize}
\end{leftcolumn}

%------------------- PRAWA KOLUMNA -------------------%
\begin{rightcolumn}
    \vspace*{-2ex} % ewentualne przesunięcie w górę, by tekst był bliżej bannera
    \noindent \fontsize{14pt}{14pt}{\textbf{\color{Violet}O MNIE:}}
    \fontsize{10pt}{10pt}
    \\ \\
    Młody specjalista w dziedzinie \textbf{uczenia maszynowego}, ze szczególnym zainteresowaniem uczeniem przez wzmacnianie.
    W swojej pracy łączę wiedzę matematyczną z algorytmiką oraz analizą danych, tworząc efektywne rozwiązania oparte na modelach ML. 
    Głównym narzędziem pracy jest dla mnie \textbf{Python}, a w projektach stawiam na solidne podstawy teoretyczne i dobrze zaprojektowaną architekturę rozwiązań.
    Od zawsze fascynują mnie algorytmy, optymalizacja i rozwiązywanie złożonych problemów przy użyciu metod matematycznych.
    \\ \\
    \rule{11cm}{1pt} \\ \\
    \fontsize{14pt}{14pt}{\textbf{\color{Violet}DOŚWIADCZENIE:}}
    \fontsize{10pt}{10pt}
    \begin{itemize}[leftmargin=*]
        \setlength{\parskip}{0pt}
        % \setlength\itemsep{0pt} 
        \item Staż w firmie \textbf{Colgate-Palmolive}. Tworzenie interaktywnej aplikacji do
        wizualizacji danych w \textbf{Pythonie}. Zastosowanie takich technik jak \\
        widzenie maszynowe (\textbf{OCR}) do rozpoznawania tekstu na obrazach. \\
        Lipiec - wrzesień 2024
        % coś więcej
        % W podpunktach
        \item Praktyki zawodowe w firmie \textbf{Zapaśnik IT}. Programowanie \\ 
        w języku \textbf{Python}. Październik - grudzień 2020
        % automatyzacja procesów w bash itd. SSH przesyłanie plików, linux
        \item Praktyki zawodowe w firmie \textbf{Sports Media}. Sieci i systemy \\
        komputerowe. Marzec - kwiecień 2020
        
        \item Koło naukowe \textbf{Robocik} działające na Politechnice Wrocławskiej. \\
        Projektowanie \textbf{sztucznej inteligencji}, pisanie algorytmów do \\
        wykrywania położenia drona podwodnego i obsługi sterowania w \\ technologii \textbf{ROS2 (Python)},
        pod zagraniczne zawody \textbf{TAC Challange}.
        % Przygotowanie, rozwój i walidacja zautomatyzowanych testów \\
        % funkcjonalności uruchamianych podczas przygotowania drona do \\
        % pływania (\textbf{Python OpenCV})
        % numerczyne rozwiązanie równania różniczkowego Friedmana \\
        % określającego ewolucję wszechświata  (\textbf{github})
        \item Członek komisji do spraw Dydaktyki i Praw Studenta
        % \item Projekt symulujący układ słoneczny (odziaływania grawitacyjne) \\ poprzez numeryczne rozwiązywanie problemu n-ciał  (\textbf{Python})
        % \item Projekt rozwiązujący numerczynie (bez użycia bibliotek) równania różniczkowe Friedmana określające ewolucję wszechświata.
        % Celem było uzyskanie informacji o wszechświecie tj. wiek, przeszły i przyszły jego rozwój (\textbf{Python})
    \end{itemize}
    \rule{11cm}{1pt} \\ \\
    \fontsize{14pt}{14pt}{\textbf{\color{Violet}WYKSZTAŁCENIE:}}
    \fontsize{10pt}{10pt}
    \begin{itemize}[leftmargin=*]
        \setlength{\parskip}{0pt}
        % \setlength\itemsep{0pt}
        \item \textbf{Informatyka - Studia Magisterskie, Politechnika Wrocławska}, marzec 2025 - obecnie
        \item \textbf{Matematyka Stosowana - Studia inżynierskie, Politechnika Wrocławska}, październik 2021 - luty 2025, \\
        - \textbf{Praca dyplomowa:} Analiza efektywności metod uczenia przez wzmacnianie w grach komputerowych
        % \item \textbf{Politechnika Wrocławska, Matematyka Stosowana, Studia \\ inżynierskie, październik 2021 - luty 2025}
        \item \textbf{Zespół Szkół Teleinformatycznych i Elektronicznych we Wrocławiu, Technikum nr 7, Technik Informatyk, wrzesień 2017 - kwiecień 2021} 
    \end{itemize}
    \rule{11cm}{1pt} \\ \\
    \fontsize{14pt}{14pt}{\textbf{\color{Violet}CERTYFIKATY:}}
    \fontsize{10pt}{10pt}
    \begin{itemize}[leftmargin=*]
        \setlength{\parskip}{0pt}
        % \setlength\itemsep{0pt} 
        \item Corporate Readiness Certificate 2024 - Data Science w praktyce
        \item Kwalifikacja EE.09 - Programowanie, tworzenie i administrowanie stronami
        internetowymi i bazami danych
        \item Kwalifikacja EE.08 - Montaż i eksploatacja systemów komputerowych, urządzeń
        peryferyjnych i sieci
    \end{itemize}
    \rule{0pt}{0pt} \\ \\ \\
    \fontsize{14pt}{14pt}{\textbf{\color{Violet}PROJEKTY:}}
    \fontsize{10pt}{10pt}
    \begin{itemize}[leftmargin=*]
        \setlength{\parskip}{0pt}
        \item \textbf{Projekt Inżynierski} - Porównanie efektywności algorytmów uczenia przez wzmacnianie w grze \textbf{Pong}. 
        Przeanalizowano dwa podejścia wykorzystujące sieci neuronowe: \textbf{Deep Q-Learning} oraz \textbf{A2C}.
        (Python, PyTorch, Gymnasium, OpenCV, NumPy)
        \item \textbf{Numeryczne rozwiązanie równania różniczkowego Friedmana}, opisującego ewolucję wszechświata – implementacja bez użycia zewnętrznych bibliotek.
        (Python)
        \item \textbf{Baza danych dla warsztatu samochodowego}, który oprócz klasycznych usług oferuje kupno, renowację i sprzedaż samochodów.
        (Python, SQL)
        \item \textbf{Symulacja graficzna grafów przepływowych} na podstawie przejazdu PKP.
        (Python, NetworkX)
        \item \textbf{Algorytm Min-Max do gry w szachy}, wzbogacony o techniki optymalizacji, takie jak Zobrist hashing.
        (Python)
        \item \textbf{Gra 2D typu Arcade}.
        (Python, Pygame)
    \end{itemize}

\end{rightcolumn}
\end{paracol}
\end{document}