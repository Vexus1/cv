\documentclass[10pt]{article}
\usepackage{polski}
\usepackage[left=1cm, right=1cm, top=0cm, bottom=1cm]{geometry}
\usepackage{graphicx}
\usepackage[T1]{fontenc}
\usepackage{charter}
\usepackage{enumitem}
\usepackage{fancyvrb}
\usepackage{nopageno}
\usepackage[dvipsnames]{xcolor}
\usepackage{tabularx} 
\usepackage{array}
\usepackage[hidelinks]{hyperref}
\newcolumntype{C}[1]{>{\centering\arraybackslash}p{#1}}
\newcommand{\longline}{\rule{19.6cm}{1pt}}
\newcolumntype{Y}{>{\centering\arraybackslash}X}
\usepackage{fontawesome5}



\begin{document}

% -- BANER tylko na pierwszej stronie --
\begin{figure}[t]
    \hskip-1.55cm\includegraphics[scale=1.3625]{Galik_BANNER.png}
\end{figure}

\vspace*{-4ex}

% Informacje kontaktowe
% \noindent
\ \ \ \ \ \ \ \ \
\faMapMarker* Wrocław, Polska \ \ \ \ \ \ \ \ \ \
\faPhone +48 663 383 000 \ \ \ \ \ \ \ \ \ \
\faEnvelope \ adrian1galik@gmail.com \ \ \ \ \ \ \ \ \ \
\faGithub \ github.com/Vexus1


\noindent \longline 
\\ \\
% O MNIE
\noindent \fontsize{14pt}{14pt}\selectfont \textbf{\color{Violet}O MNIE:}
\fontsize{10pt}{10pt}\selectfont

\noindent \longline 
\\ \\
\noindent Specjalista w dziedzinie \textbf{uczenia maszynowego}, ze szczególnym naciskiem na uczenie przez wzmacnianie.
Biegła znajomość \textbf{modeli statystycznych} oraz ich praktycznego zastosowania do \textbf{analizy danych} i rozwiązywania problemów inżynierskich.
Doświadczenie w implementacji \textbf{algorytmów numerycznych} oraz \textbf{równań różniczkowych} w zadaniach optymalizacyjnych.
W projektach stosowana jest kombinacja solidnych podstaw matematycznych, dobrze zaprojektowanej architektury oraz języka \textbf{Python} jako głównego narzędzia programistycznego. 
Przykładowe realizacje obejmują stworzenie agenta uczenia przez wzmacnianie do gry Pong, wykorzystującego dwa podejścia z użyciem sieci neuronowych: \textbf{Deep Q-Learning} (DQN) oraz \textbf{Advantage Actor-Critic} (A2C).
\\ \\
% UMIEJĘTNOŚCI TECHNICZNE
\noindent \fontsize{14pt}{14pt}\selectfont \textbf{\color{Violet}UMIEJĘTNOŚCI TECHNICZNE:}
\fontsize{10pt}{10pt}\selectfont 
\\ 
\noindent \longline 
\begin{itemize}[leftmargin=*, parsep=0.5pt]
    \item Języki programowania: \textbf{Python} (główny język 4+ lat), \textbf{SQL} (2+ lat), textbf{R} (1+ lat), \textbf{Julia} (Podstawowa wiedza)
    \item Biblioteki programistyczne: \textbf{NumPy, PyTorch, TensorFlow, Keras, Gymnasium, OpenCV, Scikit-Learn, Pandas, \\ NetworkX}
    \item Zastosowania algorytmów głębokiego uczenia maszynowego. Wysoka znajomość algorytmów uczenia przez wzmacnianie
    \item Wysoka umiejętność tworzenie modeli i zastosowania metod statystyki matematczynej wraz z wizualizacją. 
    Duża znajomość algorytmów numerycznych i ich zastoswowań. Umiejętność zastosowań równań różniczkowych w praktyce
    \item Zastosowania struktur danych: \textbf{Stosy, Kolejki, Drzewa, Grafy}
    \item Znajomość tworzenia i administrowanie stronami internetowymi: \textbf{HTML, CSS, JavaScript, React, Flask, PHP}
    \item System kontroli wersji: \textbf{Git}
    \item System operacyjny: \textbf{Linux, Windows}
    \item Powłoka systemowa UNIX: \textbf{Bash} 
    \item Wirtualizacja i izolacja środowisk: \textbf{Docker, VirtualBox}
    \item Framework wspierający rozwój oprogramowania dla robotów: \textbf{ROS2}
\end{itemize}

% DOŚWIADCZENIE
\noindent \fontsize{14pt}{14pt}\selectfont \textbf{\color{Violet}DOŚWIADCZENIE:}
\fontsize{10pt}{10pt}\selectfont

\noindent \longline 
\begin{itemize}[leftmargin=*]
    \item \textbf{Staż w firmie Colgate-Palmolive}, 07/2024 - 09/2024
    \begin{itemize}
        \item Tworzenie interaktywnej aplikacji do wizualizacji danch w \textbf{Pythonie}
        \item Zastosowanie technik widzenia maszynowego \textbf{OCR} 
        \item Analiza statystyczna danych oraz wizaulizacja na podstawie wykresów
        \item Dokumentacja techniczna dla aplikacji
    \end{itemize}
    \item \textbf{Praktyki zawodowe w firme Zapaśnik IT}, 10/2020 - 12/2020
    \begin{itemize}
        \item Tworzenie skryptów w \textbf{Bashu} 
        \item Interaktywne zarządzanie zdalnymi połączeniami: \textbf{Putty}
    \end{itemize}
    \item \textbf{Praktyki zawodowe w firmie Sports Media}, 03/2020 - 05/2020
    \begin{itemize}
        \item Zarządzanie siecami komputerowymi
        \item Tworzenie arkuszów kalkulacyjnych dla ilości i wyceny produktów: \textbf{Excel}
    \end{itemize}
\end{itemize}

% \rule{17cm}{1pt} \\
% WYKSZTAŁCENIE
\newpage
\vspace*{10pt}
\noindent \fontsize{14pt}{14pt}\selectfont \textbf{\color{Violet}WYKSZTAŁCENIE:}
\fontsize{10pt}{10pt}\selectfont 
\\ 
\noindent \longline 
\begin{itemize}[leftmargin=*]
    \item \textbf{Informatyka - Studia Magisterskie, Politechnika Wrocławska}, 03/2025 - obecnie
    \item \textbf{Matematyka Stosowana - Studia inżynierskie, Politechnika Wrocławska}, 10/2021 - 02/2025
    \begin{itemize}
    \item \textbf{Praca dyplomowa:} Analiza efektywności metod uczenia przez wzmacnianie w grach komputerowych
    \item \textbf{Kursy:} Algorytmy i struktury danych, metody numeryczne, równania różniczkowe w technice, statystyka stosowana, pakiety statystyczne, bazy danych, 
    \item \textbf{Koło naukowe KN Robocik:} Tworzenie algorytmów do wykrywania położenia drona podwodnego i obsługi sterowania w technologii \textbf{ROS2 (Python)}, pod zagraniczne zawody \textbf{TAC Challange}. 
    \item \textbf{Aktywność studencka:} Członek komisji do spraw Dydaktyki i Praw Studenta
    \end{itemize}
    \item \textbf{Zespół Szkół Teleinformatycznych i Elektronicznych we Wrocławiu, Technikum nr 7, Technik Informatyk}, \\
    09/2017 - 04/2021 
\end{itemize}

% PROJEKTY
\noindent \fontsize{14pt}{14pt}\selectfont \textbf{\color{Violet}PROJEKTY:}
\fontsize{10pt}{10pt}\selectfont
\\
\noindent \longline 
\begin{itemize}[leftmargin=*]
    \item \textbf{Projekt Inżynierski} - Porównanie efektywności algorytmów uczenia przez wzmacnianie w grze \textbf{Pong}. 
    Przeanalizowano dwa podejścia wykorzystujące sieci neuronowe: \textbf{Deep Q-Learning} oraz \textbf{A2C}.
    Projekt zawiera obszerne wprowadzenie do tematu wraz z analizą wykresów precesu uczenia. (Python, PyTorch, Gymnasium, OpenCV, NumPy)
    \item \textbf{Numeryczne rozwiązanie równania różniczkowego Friedmana} - Zastosowano numeryczne rozwiązanie równania różniczkowego bez użycia bibliotek.
    Wyliczenie wieku wrzechświata za pomocą całkowania numerycznego. Matematyczny opis projektu wykonany w notatniku Jupyter wraz z analizą techniczną. (Python)
    \item \textbf{Baza danych dla warsztatu samochodowego} - Tworzenie architektury bazy danych oraz kodu wypełnającego ją. 
    Raport zawierający analizę statystyczną danych losowych dla różnych przedmiotów oraz usług. (SQL, Python)
    % \item \textbf{Symulacja graficzna grafów przepływowych} 
    \item \textbf{Algorytm Min-Max do gry w szachy} - Algorytmiczne podejście do stworzenia bota przewidującego kilka ruchów w przód. Wykorzystanie metod tj: Zobrist Hasing, transposition table, iterative deepening. Utowrzenie GUI do gry z botem. (Python)
    \item \textbf{Gra 2D typu Arcade} - Zaprojektowana obiektowo przy użyciu biblioteki PyGame gra polegająca na zestrzeleniu \\ poruszających się przeciwników. (Python, Pygame)
\end{itemize}

% CERTYFIKATY
\noindent \fontsize{14pt}{14pt}\selectfont \textbf{\color{Violet}CERTYFIKATY:}
\fontsize{10pt}{10pt}\selectfont 
\\ 
\noindent \longline 
\begin{itemize}[leftmargin=*]
    \item Corporate Readiness Certificate 2024 - Data Science w praktyce
    \item Kwalifikacja EE.09, 2020 - Programowanie, tworzenie i administrowanie stronami internetowymi i bazami danych 
    \item Kwalifikacja EE.08, 2019 - Montaż i eksploatacja systemów komputerowych, urządzeń peryferyjnych i sieci
\end{itemize}

% JĘZYKI
\noindent \fontsize{14pt}{14pt}\selectfont \textbf{\color{Violet}JĘZYKI:}
\fontsize{10pt}{10pt}\selectfont
\\
\noindent \longline 
\begin{itemize}[leftmargin=*]
    \item Polski ojczysty
    \item Angielski C1
    \item Hiszpański A1
\end{itemize}

% ZAINTERESOWANIA
\noindent \fontsize{14pt}{14pt}\selectfont \textbf{\color{Violet}ZAINTERESOWANIA:}
\fontsize{10pt}{10pt}\selectfont
\\
\noindent \longline 
\begin{itemize}[leftmargin=*]
    \item Uczenie maszynowe
    \item Matematyka
    \item Astrofizyka
\end{itemize}

\end{document}
