\documentclass[10pt]{article}
\usepackage{polski}
\usepackage[left=1cm, right=1.5cm, top=0cm, bottom=1cm]{geometry}
\usepackage{graphicx}
\usepackage[T1]{fontenc}
\usepackage{charter}
\usepackage{enumitem}
\usepackage{fancyvrb}
\usepackage{nopageno}
\usepackage[dvipsnames]{xcolor}
\usepackage{tabularx} 
\usepackage{array}
\newcolumntype{C}[1]{>{\centering\arraybackslash}p{#1}}

\begin{document}

% -- BANER tylko na pierwszej stronie --
\begin{figure}[t]
    \hskip-1.55cm\includegraphics[scale=1.3625]{Galik_BANNER.png}
\end{figure}

\vspace*{-4ex}

% Informacje kontaktowe
\noindent
\begin{tabularx}{\textwidth}{l C{12cm} r}
    \textbf{+48 663 383 000} &
    \textbf{adrian1galik@gmail.com} &
    \textbf{github.com/Vexus1}
\end{tabularx} \\
\rule{\textwidth}{1pt} \\

% O MNIE
\noindent \fontsize{14pt}{14pt}\selectfont \textbf{\color{Violet}O MNIE:}
\fontsize{10pt}{10pt}\selectfont

\noindent Młody specjalista w dziedzinie \textbf{uczenia maszynowego}, ze szczególnym zainteresowaniem uczeniem przez wzmacnianie.
Posiadam wysoką znajomość \textbf{modeli statystycznych} oraz ich praktycznych zastosowań.
W swojej pracy łączę wiedzę matematyczną z algorytmiką oraz analizą danych, tworząc efektywne rozwiązania oparte na modelach ML. 
Dysponuję również umiejętnościami w zakresie \textbf{algorytmów numerycznych} i \textbf{równań różniczkowych}, które wykorzystuję do rozwiązywania problemów inżynierskich i optymalizacyjnych. 
Głównym narzędziem pracy jest dla mnie \textbf{Python}, a w projektach stawiam na solidne podstawy teoretyczne i dobrze zaprojektowaną architekturę rozwiązań.
Od zawsze fascynują mnie algorytmy, optymalizacja i rozwiązywanie złożonych problemów przy użyciu metod matematycznych.

\noindent \rule{\textwidth}{1pt} \\

% UMIEJĘTNOŚCI TECHNICZNE
\noindent \fontsize{14pt}{14pt}\selectfont \textbf{\color{Violet}UMIEJĘTNOŚCI TECHNICZNE:}
\fontsize{10pt}{10pt}\selectfont
\begin{itemize}[leftmargin=*]
    \item \textbf{Python} jako główny język programowania z wysokim poziomem znajomości 
    \item Biblioteki programistyczne: \textbf{NumPy, PyTorch, TensorFlow, Keras, Gymnasium, OpenCV, Scikit-Learn, Pandas, \\ NetworkX}
    \item Zastosowania algorytmów głębokiego uczenia maszynowego
    \item Wysoka znajomość algorytmów uczenia przez wzmacnianie
    \item Wysoka umiejętność tworzenie modeli i zastosowania metod statystyki matematczynej wraz z wizualizacją
    \item Duża znajomość algorytmów numerycznych i ich zastoswowań
    \item Wysoka umiejętność zastosowań równań różniczkowych w praktyce
    \item Umiejętność zastosowania struktur danych: \textbf{Stosy, Kolejki, Drzewa, Grafy}
    \item Zarządzanie bazami danych: \textbf{SQL} wraz z automatyzacją za pomocą \textbf{Pythona}
    \item Znajomość tworzenia i administrowanie stronami internetowymi: \textbf{HTML, CSS, JavaScript, React, Flask, PHP}
    \item System kontroli wersji: \textbf{Git}
    \item System operacyjny: \textbf{Linux, Windows}
    \item Znajomość nowego języka do obliczeń numerycznych: \textbf{Julia}
    \item Powłoka systemowa UNIX: \textbf{Bash} 
    \item Wirtualizacja i izolacja środowisk: \textbf{Docker, VirtualBox}
    \item Framework wspierający rozwój oprogramowania dla robotów: \textbf{ROS2}
\end{itemize}
\rule{17cm}{1pt} \\

% JĘZYKI
\fontsize{14pt}{14pt}\selectfont \textbf{\color{Violet}JĘZYKI:}
\fontsize{10pt}{10pt}\selectfont
\begin{itemize}[leftmargin=*]
    \item Polski ojczysty
    \item Angielski C1
    \item Hiszpański A1
\end{itemize}
\rule{17cm}{1pt} \\

% DOŚWIADCZENIE
\fontsize{14pt}{14pt}\selectfont \textbf{\color{Violet}DOŚWIADCZENIE:}
\fontsize{10pt}{10pt}\selectfont
\begin{itemize}[leftmargin=*]
    \item Staż w firmie \textbf{Colgate-Palmolive}. Tworzenie interaktywnej aplikacji do wizualizacji danych w \textbf{Pythonie}. Zastosowanie takich technik jak widzenie maszynowe (\textbf{OCR}) do rozpoznawania tekstu na obrazach. \\ Lipiec - wrzesień 2024
    \item Praktyki zawodowe w firmie \textbf{Zapaśnik IT}. Programowanie \\ w języku \textbf{Python}. Październik - grudzień 2020
    \item Praktyki zawodowe w firmie \textbf{Sports Media}. Sieci i systemy \\ komputerowe. Marzec - kwiecień 2020
    \item Koło naukowe \textbf{Robocik} działające na Politechnice Wrocławskiej. \\ Projektowanie \textbf{sztucznej inteligencji}, pisanie algorytmów do wykrywania położenia drona podwodnego i obsługi sterowania w \\ technologii \textbf{ROS2 (Python)}, pod zagraniczne zawody \textbf{TAC Challange}.
    \item Członek komisji do spraw Dydaktyki i Praw Studenta
\end{itemize}
\rule{17cm}{1pt} \\

% WYKSZTAŁCENIE
\fontsize{14pt}{14pt}\selectfont \textbf{\color{Violet}WYKSZTAŁCENIE:}
\fontsize{10pt}{10pt}\selectfont
\begin{itemize}[leftmargin=*]
    \item \textbf{Informatyka - Studia Magisterskie, Politechnika Wrocławska}, marzec 2025 - obecnie
    \item \textbf{Matematyka Stosowana - Studia inżynierskie, Politechnika Wrocławska}, październik 2021 - luty 2025, \\
    - \textbf{Praca dyplomowa:} Analiza efektywności metod uczenia przez wzmacnianie w grach komputerowych
    \item \textbf{Zespół Szkół Teleinformatycznych i Elektronicznych we Wrocławiu, Technikum nr 7, Technik Informatyk, wrzesień 2017 - kwiecień 2021} 
\end{itemize}
\rule{17cm}{1pt} \\

% CERTYFIKATY
\fontsize{14pt}{14pt}\selectfont \textbf{\color{Violet}CERTYFIKATY:}
\fontsize{10pt}{10pt}\selectfont
\begin{itemize}[leftmargin=*]
    \item Corporate Readiness Certificate 2024 - Data Science w praktyce
    \item Kwalifikacja EE.09 - Programowanie, tworzenie i administrowanie stronami internetowymi i bazami danych
    \item Kwalifikacja EE.08 - Montaż i eksploatacja systemów komputerowych, urządzeń peryferyjnych i sieci
\end{itemize}
\rule{17cm}{1pt} \\

% PROJEKTY
\fontsize{14pt}{14pt}\selectfont \textbf{\color{Violet}PROJEKTY:}
\fontsize{10pt}{10pt}\selectfont
\begin{itemize}[leftmargin=*]
    \item \textbf{Projekt Inżynierski} - Porównanie efektywności algorytmów uczenia przez wzmacnianie w grze \textbf{Pong}. 
    Przeanalizowano dwa podejścia wykorzystujące sieci neuronowe: \textbf{Deep Q-Learning} oraz \textbf{A2C}. \\
    (Python, PyTorch, Gymnasium, OpenCV, NumPy)
    \item \textbf{Numeryczne rozwiązanie równania różniczkowego Friedmana}, opisującego ewolucję wszechświata – implementacja bez użycia zewnętrznych bibliotek. \\
    (Python)
    \item \textbf{Baza danych dla warsztatu samochodowego}, który oprócz klasycznych usług oferuje kupno, renowację i sprzedaż samochodów. \\
    (Python, SQL)
    \item \textbf{Symulacja graficzna grafów przepływowych} na podstawie przejazdu PKP. \\
    (Python, NetworkX)
    \item \textbf{Algorytm Min-Max do gry w szachy}, wzbogacony o techniki optymalizacji, takie jak Zobrist hashing. \\
    (Python)
    \item \textbf{Gra 2D typu Arcade}. \\
    (Python, Pygame)
\end{itemize}
\rule{17cm}{1pt} \\

% ZAINTERESOWANIA
\fontsize{14pt}{14pt}\selectfont \textbf{\color{Violet}ZAINTERESOWANIA:}
\fontsize{10pt}{10pt}\selectfont
\begin{itemize}[leftmargin=*]
    \item Uczenie maszynowe
    \item Matematyka
    \item Astrofizyka
\end{itemize}

\end{document}
